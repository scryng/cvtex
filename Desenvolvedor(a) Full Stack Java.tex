%-------------------------
% Resume in Latex
% Author : Jake Gutierrez
% Based off of: https://github.com/sb2nov/resume
% License : MIT
%------------------------

\documentclass[letterpaper,11pt]{article}

\usepackage{latexsym}
\usepackage[empty]{fullpage}
\usepackage{titlesec}
\usepackage{marvosym}
\usepackage[usenames,dvipsnames]{color}
\usepackage{verbatim}
\usepackage{enumitem}
\usepackage[hidelinks]{hyperref}
\usepackage{fancyhdr}
\usepackage[english]{babel}
\usepackage{tabularx}
\usepackage{fontawesome5}
\usepackage{multicol}
\setlength{\multicolsep}{-3.0pt}
\setlength{\columnsep}{-1pt}
\input{glyphtounicode}


%----------FONT OPTIONS----------
% sans-serif
% \usepackage[sfdefault]{FiraSans}
% \usepackage[sfdefault]{roboto}
% \usepackage[sfdefault]{noto-sans}
% \usepackage[default]{sourcesanspro}

% serif
% \usepackage{CormorantGaramond}
% \usepackage{charter}


\pagestyle{fancy}
\fancyhf{} % clear all header and footer fields
\fancyfoot{}
\renewcommand{\headrulewidth}{0pt}
\renewcommand{\footrulewidth}{0pt}

% Adjust margins
\addtolength{\oddsidemargin}{-0.6in}
\addtolength{\evensidemargin}{-0.5in}
\addtolength{\textwidth}{1.19in}
\addtolength{\topmargin}{-.7in}
\addtolength{\textheight}{1.4in}

\urlstyle{same}

\raggedbottom
\raggedright
\setlength{\tabcolsep}{0in}

% Sections formatting
\titleformat{\section}{
  \vspace{-4pt}\scshape\raggedright\large\bfseries
}{}{0em}{}[\color{black}\titlerule \vspace{-5pt}]

% Ensure that generate pdf is machine readable/ATS parsable
\pdfgentounicode=1

%-------------------------
% Custom commands
\newcommand{\resumeItem}[1]{
  \item\small{
    {#1 \vspace{-2pt}}
  }
}

\newcommand{\classesList}[4]{
    \item\small{
        {#1 #2 #3 #4 \vspace{-2pt}}
  }
}

\newcommand{\resumeSubheading}[4]{
  \vspace{-2pt}\item
    \begin{tabular*}{1.0\textwidth}[t]{l@{\extracolsep{\fill}}r}
      \textbf{#1} & \textbf{\small #2} \\
      \textit{\small#3} & \textit{\small #4} \\
    \end{tabular*}\vspace{-7pt}
}

\newcommand{\resumeSubSubheading}[2]{
    \item
    \begin{tabular*}{0.97\textwidth}{l@{\extracolsep{\fill}}r}
      \textit{\small#1} & \textit{\small #2} \\
    \end{tabular*}\vspace{-7pt}
}

\newcommand{\resumeProjectHeading}[2]{
    \item
    \begin{tabular*}{1.001\textwidth}{l@{\extracolsep{\fill}}r}
      \small#1 & \textbf{\small #2}\\
    \end{tabular*}\vspace{-7pt}
}

\newcommand{\resumeSubItem}[1]{\resumeItem{#1}\vspace{-4pt}}

\renewcommand\labelitemi{$\vcenter{\hbox{\tiny$\bullet$}}$}
\renewcommand\labelitemii{$\vcenter{\hbox{\tiny$\bullet$}}$}

\newcommand{\resumeSubHeadingListStart}{\begin{itemize}[leftmargin=0.0in, label={}]}
\newcommand{\resumeSubHeadingListEnd}{\end{itemize}}
\newcommand{\resumeItemListStart}{\begin{itemize}}
\newcommand{\resumeItemListEnd}{\end{itemize}\vspace{-5pt}}

%-------------------------------------------
%%%%%%  RESUME STARTS HERE  %%%%%%%%%%%%%%%%%%%%%%%%%%%%

\usepackage{geometry}
\geometry{left=1in, right=1in, top=1in, bottom=1in}
\setlength{\columnsep}{1pt}

\begin{document}

%----------CABEÇALHO----------
\begin{center}
    {\Huge \scshape Gustavo Sousa} \\ \vspace{1pt}
    São Paulo, Brasil \\ \vspace{1pt}
    \small \raisebox{-0.1\height}\faPhone\ (15) 99194-8282 ~ 
    \href{mailto:gnyrcs@gmail.com}{\raisebox{-0.2\height}\faEnvelope\  \underline{gnyrcs@gmail.com}} ~ 
    \href{https://linkedin.com/in/scryng}{\raisebox{-0.2\height}\faLinkedin\ \underline{linkedin.com/in/scryng}}  ~
    \href{https://github.com/scryng}{\raisebox{-0.2\height}\faGithub\ \underline{github.com/scryng}}
    \vspace{-8pt}
\end{center}

%-----------EDUCAÇÃO-----------
\section{Educação}
  \resumeSubHeadingListStart
    \resumeSubheading
      {Senai São Paulo}{Jan. 2024 -- Dez. 2025}
      {Tecnólogo em Análise e Desenvolvimento de Sistemas}{Sorocaba, São Paulo}
    \resumeSubheading
      {Senai São Paulo}{Ago. 2019 -- Jul. 2021}
      {Técnico em Eletroeletrônica}{Sorocaba, São Paulo}
  \resumeSubHeadingListEnd

%-----------CURSOS RELEVANTES-----------
\section{Conhecimentos Relevantes}
    \begin{multicols}{2}
        \begin{itemize}[itemsep=-5pt, parsep=3pt]
            \item\small Desenvolvimento Full Stack
            \item Engenharia de Software e Metodologias
            \item Estruturas de Dados e Algoritmos
            \item Computação em Nuvem (AWS, GCP, Cisco)
            \item Desenvolvimento de Soluções Móveis e Web
            \item Big Data e Análise de Dados
            \item Integrações entre TI, Automação e IIoT
            \item Protocólos de Comunicação Industrial (MQTT, OPC UA)
            \item Ferramentas e Conceitos da Indústria 4.0
            \item Análise de Riscos em Cibersegurança
        \end{itemize}
    \end{multicols}
    \vspace*{2.0\multicolsep}

%-----------EXPERIÊNCIA-----------
\section{Experiência Profissional}
  \resumeSubHeadingListStart

    \resumeSubheading
      {2RP Net}{Mar. 2024 -- Presente - 1 ano}
      {Desenvolvedor Back-End}{Brasil (Remoto)}
      \resumeItemListStart
        \resumeItem{Atuo em soluções robustas utilizando Python e melhores práticas no desenvolvimento de scripts de alto desempenho utilizando LLM/IA para resultados dinâmicos e personalizados a fim de melhorar migrações e extrações de dados para o Google Cloud Platform (GCP) para grandes empresas, adotando documentação técnica com DocString, Markdown e Mermaid para fácil manutenção e escalabilidade futura entre a equipe.}
        \resumeItem{Implementação de padrões de commit e gerenciamento de branches para versionamento de código de scripts, juntamente com adaptações do sistema para observabilidade e escalabilidade.}
      \resumeItemListEnd
      
    \resumeSubheading
      {Scryng}{Fev. 2022 -- Fev. 2025 - 3 anos}
      {Desenvolvedor Lua}{Brasil (Remoto)}
      \resumeItemListStart
        \resumeItem{Desenvolvimento de Back-End para scripts em motores de jogos utilizando a linguagem de programação Lua.}
        \resumeItem{Controle de versão de todos os scripts do projeto com Git, GitHub e GitFlow, utilizando gerenciamento de branches e padrões de commit.}
        \resumeItem{Trabalho em um ambiente dinâmico e acelerado, com entregas contínuas.}
        \resumeItem{Otimização e depuração de código Lua para melhorar a funcionalidade do sistema e reduzir erros.}
      \resumeItemListEnd

  \resumeSubHeadingListEnd
\vspace{-16pt}

%-----------PROJETOS-----------
\section{Projetos}
    \vspace{-5pt}
    \resumeSubHeadingListStart
      \resumeProjectHeading
        {\textbf{\href{https://github.com/scryng/tag_sstem}{\raisebox{-0.2\height}\faGithub\ \underline{TAG System}}} $|$ \emph{CSharp, ASP.NET, React, Next.js, Typescript, Markdown, Mermaid}}{Dez. 2024} \\ 
        \resumeItemListStart
            \resumeItem{Sistema IIoT de controle de acesso utilizando ESP32 e RFID, integrado a uma aplicação Web para monitoramento e gerenciamento em tempo real.}
            \resumeItem{Leitura de RFID, autenticação dos cartões cadastrados, monitoramento em tempo real com dashboards e relatórios.}
        \resumeItemListEnd
        \vspace{-13pt}
      \resumeProjectHeading
        {\textbf{\href{https://github.com/scryng/coffee_shop_mvc}{\raisebox{-0.2\height}\faGithub\ \underline{Coffeeshop MVC}}} $|$ \emph{CSharp, ASP.NET, Git, Markdown, Markdown}}{Dez. 2024} \\ 
        \resumeItemListStart
            \resumeItem{Criação da Web API com ASP.NET, Scanfolding (gerador de código) e Entity Core Framework.}
            \resumeItem{Realização de testes utilizando Swagger, Postman e Powershell.}
            \resumeItem{Implementação de Data Transfer Object (DTO) para proteger contra postagens excessivas e/ou dados sensíveis na Web API.}
        \resumeItemListEnd
        \vspace{+13pt}
      \resumeProjectHeading
        {\textbf{\href{https://github.com/scryng/queue_challenge_for_5v5_matches}{\raisebox{-0.2\height}\faGithub\ \underline{Sistema de Fila para Partidas PvP 5v5}}} $|$ \emph{Lua, Git}}{Nov. 2024} \\ 
        \resumeItemListStart
            \resumeItem{Desenvolvi um sistema de fila que identifica grupos de jogadores buscando uma partida e forma equipes prontas para solicitar a criação de uma partida do modo de jogo.}
            \resumeItem{Os grupos de jogadores persistem após o fim da partida, com os jogadores retornando para buscar novas equipes e/ou partidas sem desmantelar o grupo.}
        \resumeItemListEnd
        \vspace{-13pt}
      \resumeProjectHeading
        {\textbf{\href{https://github.com/scryng/apis_banking_ecosystem}{\raisebox{-0.2\height}\faGithub\ \underline{Ecossistema de APIs Bancárias}}} $|$ \emph{Python, Poetry, RabbitMQ, Linux, Docker e k8s, Git}}{Set. 2024} \\ 
        \resumeItemListStart
            \resumeItem{Criação de microsserviços para compor um ecossistema bancário contendo uma aplicação webhook, uma aplicação de armazenamento de dados, uma aplicação de streaming para produtos, um banco de dados SQL, um banco de dados NoSQL e um serviço de mensagens (RabbitMQ e Pub/Sub).}
            \resumeItem{Implementação de um ecossistema capaz de criptografar dados de eventos de três tipos diferentes, enviá-los para um tópico de mensagens para gerenciar a fila, e redirecionar esses dados para a API de armazenamento no banco de dados, permitindo que uma API de streaming consuma os dados.}
            \resumeItem{Uso de Load Balance para lidar com escalabilidade e containers Docker para facilitar a avaliação e os testes das aplicações.}
        \resumeItemListEnd
        \vspace{-13pt}
      \resumeProjectHeading
        {\textbf{\href{https://github.com/scryng/game_pacman_java}{\raisebox{-0.2\height}\faGithub\ \underline{Jogo Pacman em Java}}} $|$ \emph{Java, Java 8, JavaFX, IntelliJ IDEA, Git}}{Jul. 2024} \\ 
          \resumeItemListStart
            \resumeItem{Criação de um jogo Pacman utilizando Java e Programação Orientada a Objetos, explorando suas vantagens como classes abstratas, herança, polimorfismo, classes estáticas e interfaces.}
          \resumeItemListEnd 
          \vspace{-13pt}
      \resumeProjectHeading
        {\textbf{\href{https://github.com/scryng}{\raisebox{-0.2\height}\faGithub\ \underline{FiveM City}}} $|$ \emph{Lua, vRP, SQL, Computação em Nuvem, Git, LLM}}{Jul. 2022} \\ 
        \resumeItemListStart
            \resumeItem{Desenvolvimento de uma cidade para GTA V Roleplay com vários sistemas simulando a vida real.}
            \resumeItem{Melhoria e refatoração de todo o padrão de banco de dados e estruturação de funções do Framework para garantir um servidor de alto nível.}
            \resumeItem{Hospedagem, versionamento e entrega contínua da cidade para os jogadores, trazendo atualizações e melhorias recorrentes.}
        \resumeItemListEnd
    \resumeSubHeadingListEnd
\vspace{-15pt}


%-----------SKILLS-----------
\section{Habilidades Técnicas}
 \begin{itemize}[leftmargin=0.15in, label={}]
    \small{\item{
     \textbf{Linguagens}{: Lua, TypeScript, Python, CSharp, Java, SQL, C++} \\ 
     \textbf{Frameworks e Ferramentas}{: vRP, Nest, FastAPI, ASP.NET, Spring} \\ 
     \textbf{Banco de dados}{: SQLite, PostgreSQL, MongoDB, MariaDB} \\ 
     \textbf{Cloud e DevOps}{: Docker and k8s, Linux, Git, AWS, GCP} \\ 
     \textbf{Metodologias}{: Agile, Scrum, CI/CD, System Design, Design Patterns} \\ 
    }}
 \end{itemize}
 \vspace{-16pt}

%-----------CERTIFICATIONS-----------
\section{Certificações}
  \resumeSubHeadingListStart
 \begin{itemize}[leftmargin=0.15in, label={}]
        \resumeItem{Introduction to Data Analytics on Google Cloud}
        \resumeItem{Analyze Speech and Language with Google APIs Skill Badge}
        \resumeItem{Build a Secure Google Cloud Network Skill Badge}
        \resumeItem{Build and Deploy Machine Learning Solutions on Vertex AI Skill Badge}
        \resumeItem{Create ML Models with BigQuery ML Skill Badge}
        \resumeItem{Google Cloud Computing Foundations Certificate}
        \resumeItem{Google Cloud Computing Foundations: Data, ML, and AI in Google Cloud}
        \resumeItem{Google Cloud Computing Foundations: Infrastructure in Google Cloud}
        \resumeItem{Google Cloud Computing Foundations: Networking}
        \resumeItem{Introduction to AI and Machine Learning on Google Cloud}
        \resumeItem{Prepare Data for ML APIs on Google Cloud Skill Badge}
        \resumeItem{Set Up an App Dev Environment on Google Cloud Skill Badge}
        \resumeItem{Use Machine Learning APIs on Google Cloud Skill Badge}
        \resumeItem{Implement Load Balancing on Compute Engine Skill Badge}
    \end{itemize}
  \resumeSubHeadingListEnd
\vspace{-16pt}

%-----------LANGUAGES-----------
\section{Idiomas}
  \resumeSubHeadingListStart
    \resumeItem{Inglês - Proficiência profissional (B2)}
    \resumeItem{Espanhol - Conhecimento básico}
  \resumeSubHeadingListEnd
\vspace{-16pt}

\end{document}
